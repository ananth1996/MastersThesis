\chapter{Introduction}



In recent years, researchers have become increasingly interested in understanding the behaviour of voters in social networks. For example, voting for a common time slot in Doodle polls \cite{zou2015strategicDoodle}, passing bills in Unites States Congress \cite{karimi2019multicongress} or electing administrators in Wikipedia \cite{jankowski-lorek2013MBSN,cabunducan2011voting,lee2012uncovering}. Knowledge of the factors that motivate voters is of great importance in selecting successful polices or candidates. When these factors are derived from the network of voters, we have a greater insight into the interactions and effect influence in a community. 


CARS model

Move 1: Establish a territory:\\
Step1: Claim centrality\\
Mention the importance of elections  and how they shape societies\\

Step2: Making Generalization:\\
Well documented voting processes in US Congress, EU parliament and Wikipedia.\\

Step3: Reviewing previous research\\
How previous models focus on predicting results and judgements.\\
Use graph based information and text to get features.   \\
Model as a link prediction problem in online networks.\\

Move2:Establishing a niche\\

Step1: Counter claiming\\
How link prediction doesn't account for the differences in voting patters.\\

Step2: Indicating a Gap\\
How the voter behaviour cannot be directly analyzed using graph theory concepts.\\
Model interpretation is weaker.\\



\section{Thesis Contribution}
The Local Signed network model that utilizes interaction between voters to predict their votes. 

\section{Thesis Outline}
