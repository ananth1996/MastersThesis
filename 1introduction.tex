\chapter{Introduction}



In recent years, researchers have become increasingly interested in understanding the behaviour of voters in social networks. Knowledge of the factors that motivate voters is of great importance in selecting successful policies or candidates. This is a classic problem and has been studied in the fields of game theory and political science \cite{zou2015strategicDoodle,kearns2009behavioral,tal2015a}. More recently, there is a focus on using information from the network of voters to model their behaviour. This provides an insight into the interactions and effect of influence on voters in a community. For example, voting for bills in the United States Congress \cite{karimi2019multicongress} or electing administrators in Wikipedia \cite{jankowski-lorek2013MBSN,cabunducan2011voting,lee2012uncovering}.

Votes can be represented as a \textit{signed} network with positive or negative links. Finding groups using clustering techniques \cite{brito2020aBrazil,levorato2016brazilian,chiang2014prediction} and predicting signed links \cite{leskovec2010predicting,leskovec2010signed,chiang2011exploiting} in these networks is well researched. These approaches provide an ability to understand the group dynamics at play and predict votes and in such a network. However, they do not consider the iterative and chronological nature of the voting that takes place in these networks. In cases where research does focus on voter models, they rely on external features to build machine learning models that are task-specific and static \cite{karimi2019multicongress,jankowski-lorek2013MBSN}.

In this thesis, we propose a model that utilizes the local signed network of a voter and can be iteratively trained. It will predict the vote, which when added to the network will comply the most with concepts of balance and status for signed networks. After all the votes are cast in a session, the model can be easily updated to improve quality and is, therefore, dynamic. 
The results show that our model outperforms machine learning based models and traditional signed link prediction solutions.   
\section{Thesis Outline}
The rest of the thesis is organized in the following manner. We discuss the background relating to signed graphs, hierarchy in directed networks in Section \ref{chp:graph-theory}.