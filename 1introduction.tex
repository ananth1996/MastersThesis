\chapter{Introduction}



In recent years, researchers have become increasingly interested in understanding the behaviour of voters in social networks. Knowledge of the factors that motivate voters is of great importance in selecting successful policies or candidates. This is a classic problem and has been studied in the fields of game theory and political science \cite{zou2015strategicDoodle,kearns2009behavioral,tal2015a}. More recently, there is a focus on using information from the network of voters to model their behaviour. This provides an insight into the interactions and effect of influence on voters in a community. For example, voting for bills in the United States Congress \cite{karimi2019multicongress} or electing administrators in Wikipedia \cite{jankowski-lorek2013MBSN,cabunducan2011voting,lee2012uncovering}.

Votes can be represented as a \textit{signed} network with positive or negative links. Finding groups using clustering techniques \cite{brito2020aBrazil,levorato2016brazilian,chiang2014prediction} and predicting signed links \cite{leskovec2010predicting,leskovec2010signed,chiang2011exploiting} in these networks is well researched. These approaches provide an ability to understand the group dynamics at play and predict votes and in such a network. However, they do not consider the iterative and chronological nature of the voting that takes place in these networks.



Step1: Counter claiming\\
How link prediction doesn't account for the differences in voting patters.\\

Step2: Indicating a Gap\\
How the voter behaviour cannot be directly analyzed using graph theory concepts.\\
Model interpretation is weaker.\\



\section{Thesis Contribution}
The Local Signed network model that utilizes interaction between voters to predict their votes. 

\section{Thesis Outline}
