\chapter{Graph Theory}
\label{chp:graph-theory}
In this chapter we will provide the fundamentals of the graph theory concepts required to understand the rest of the thesis. In Section~\ref{sec:prelim} we cover the basic definitions, terminologies used to describe different types of graphs. Then we define a signed graph and discuss its unique properties in Section~\ref{sec:signed-graphs}. We outline the theory of balance in signed networks and methods to measure it in Section~\ref{sec:balance-theory}.Next, we discuss the theory of status and illustrate the differences to balance in Section~\ref{sec:status-theory}. Lastly, we explain techniques of finding hierarchies in directed networks and the concept of agony in Section~\ref{sec:hierarchy}.

\section{Preliminaries}
\label{sec:prelim}
In this section we define the various types of graphs and their basic properties. The notation and terminologies used closely follows those used in Diestel~\cite{diestel1997graph}. Graphs are structures that describe relationships between entities. These entities are called \textit{vertices} and entities related to one another are joined by edges. The terms graph, vertices and edges can be used interchangeably with \textit{network}, \textit{nodes} and \textit{links} respectively.

Graphs can be classified broadly into two types based on whether the edges posses a direction or not. We now go on to define them in detail.
\subsection{Undirected Graphs}
An undirected graph is pair $G=(V,E)$, where $V$ is the set of vertices and $E$ is the set $E \subseteq \{ (u,v) \mid u,v \in V\}$ of unordered pairs of vertices called edges. In this thesis we will deal with only \textit{simple graphs}, i.e. no are so self loops $(u,v)\in V \times V, ~ u\neq v$ and there is at most one edge between vertices $u$ and $v$. 

The number of the vertices in a graph is called the \textit{order} of the graphs is denoted by $n= |G|$ and the \textit{size} of a graph is the number of edges denoted by $m = \|G\|$ or $m=|E|$. A vertex $u$ is \textit{adjacent} to $v$ is they are the end points of an edge, $(u,v) \in E$. All the vertices adjacent to vertex $v$ is called the \textit{neighbourhood} of $v$ and is denoted by $N(v)$. The \textit{degree} of a vertex $v$ is the number of nodes adjacent to that vertex and is denoted by $d(v) = |N(v)|$. 

The edges of an undirected graph can also have an associated value. This value can indicate the distance or similarity between a pair of vertices. These values are called \textit{weights} and the corresponding graph is called a \textit{weighted undirected graph}. Therefore, a weighted graph is defined as a triple $G=(V,E,w)$, where $w:E \rightarrow \mathbb{R}^{+}$ is a function that maps an edge $e$ to a positive real weight $w(e)$. Now an \textit{unweighed graph} is simply a weighted graph where the function $w$ is defined as: if $e \in E$ then $w(e)=1$ else $w(e)=0$. The degree of a a vertex $v$ in a weighed graph is the sum of the weights to all the neighbours of $v$ and is defined as $d(v) = \sum_{u\in N(v)}w((u,v))$ An example of a weighted undirected graph is shown in Figure~\ref{fig:weighted-undirected}. 
\begin{figure}[!ht]
    \centering
    \tikzset{
    position/.style args={#1:#2 from #3}{
        at=(#3.#1), anchor=#1+180, shift=(#1:#2)
    }
}

\begin{tikzpicture}

    \begin{scope}[every node/.style={circle,thick,draw}]
        \node (v1) at (0,0) {$v_1$};
        \node[below=1cm of v1] (v2) {$v_2$};
        \node[position=45:2cm from v2] (v3) {$v_3$};
        \node[position=150:2cm from v1] (v4) {$v_4$};
        \node[left=3cm of v2] (v5) {$v_5$};
        \node[position=45:1cm from v5] (v6) {$v_6$};
    \end{scope}

    \begin{scope}[>={Stealth[red]},
        every edge/.style={thick,draw},
        every node/.style={fill=white,circle}]
        % \draw (v4) -- (v1) -- (v3) -- (v2) -- (v1);
        \path (v1) edge (v2) 
            (v2) edge (v3)
            (v1) edge node[above] {$1$} (v3)
            (v1) edge (v4)
            (v5) edge (v6)
            ;
    \end{scope}
  \end{tikzpicture}
  
    \caption{An example of a weighted undirected graph}
    \label{fig:weighted-undirected}
\end{figure}


\subsection{Directed Graphs}
The main distinction regarding a \textit{directed graph} (or \textit{digraph}) is that the edges are ordered pairs, i.e.$(u,v) \neq (v,u)$. Therefore, a directed graph has a similar definition: a pair $G=(V,E)$, where $V$ is the set of vertices and $E$ is the set of \textit{ordered} pairs of vertices. Now given an edge $e=(u,v)$ we can define a source function $\src:E\rightarrow V$ such that $\src(e)=u$ and a destination function $d:E\rightarrow V$ where $\dest(e)=v$. These functions classify the vertices in an edge $e$ as either the source or the destination. In this thesis, we deal only with \textit{simple directed graphs}, i.e. no self-loops and there can be at most one edge from $u$ to $v$. 

As the edges now have an inherent direction we can define the \textit{successors} and \textit{predecessors} of a node $v$. A vertex $u$ is called the \textit{successor} of a node $v$ is there exists a directed edge from $v$ to $u$, therefore the set of successors for a vertex $v$ can be defined as $S(v) = \{u \mid (v,u) \in E\}$. A \textit{predecessor} of a node $v$ is a vertex $u$ such that there exists a directed edge from $u$ to $v$, the set of predecessors for a vertex $v$ can de defined as $P(v) = \{u \mid (u,v) \in E\}$. We now a vertex $u$ that is either a successor or a predecessor of a vertex $v$ can be called a neighbour of the vertex $v$. Therefore, we define the \textit{neighbourhood} of a vertex $v$ as the set of vertices in the union of successors and predecessor, i.e $N(v) = S(u) \cup P(v)$. This definition is also compatible with undirected graphs because if $(u,v) \in E$ then $(v,u) \in E$. 

Directed graphs can also have values associated with each directed edge called a \textit{weight}. A \textit{weighted directed graph} can be defined as a triple $G=(V,E,w)$, where the weight function $w:E \rightarrow \mathbb{R}^{+}$ that maps each edge $e$ to a weight $w(e)$. The indegree of a vertex $v$ is defined as the sum of the edge weights from the predecessors of $v$ and is denoted as $\indegree(v) = \sum_{u \in P(v)} w((u,v))$. Similarly, the outdegree of a vertex $v$ is defined as the sum of the edge weights to the successors of $v$ and is denoted by $\outdegree(v) = \sum_{u \in S(v)}w((v,u))$.
Figure~\ref{fig:weighted-directed} shows an example of a weighted directed graph.

\begin{figure}[!ht]
    \centering
    \input{images/directed-graph.tex}
    \caption{An example of a weighted directed graph}
    \label{fig:weighted-directed}
\end{figure}

\section{Signed Graphs}
\label{sec:signed-graphs}


\section{Balance Theory}
\label{sec:balance-theory}
\begin{itemize}
    \item Explain balance theory origin and significance.
    \item Illustrate with triads and examples
    \item Define mathematical background to measure balance through the Eigen decomposition of the graph Laplacian  
\end{itemize}

\section{Status Theory}
\label{sec:status-theory}
\begin{itemize}
    \item Describe the nature of the directed setting
    \item Illustrate the differences to Balance theory
    \item Mention existing ways to measure violation to status in a network
\end{itemize}

\section{Hierarchy in directed networks}
\label{sec:hierarchy}
\begin{itemize}
    \item Discuss the hierarchy in DAGs
    \item Explain concept of Agony
    \item Provide existing algorithms to find the ranking of nodes.
\end{itemize}