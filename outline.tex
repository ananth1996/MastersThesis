\chapter{Experiments}
\label{chp:experiments}
In this section we first describe the datasets that will be used in building our vote prediction models. Then we discuss the various linear and graphical models that we consider and their implementations details. Lastly we define the metrics and other means of evaluating the models and the results.

\section{Datasets}
\begin{itemize}
    \item Maybe a short description of existing SNAP datasets and their limitations
    \item The details of the \textit{Wiki-RfA} data and the \textit{User-Contribution} datasets
\end{itemize}
\section{Graphs}
\begin{itemize}
    \item Discuss the process of extraction of the various graphs discussed in the previous sections
    \item \textbf{Agree Graphs and Follows Graph}, where we measure the degree to which one user agrees and follows another user in previous elections
    \item \textbf{Topic similarity} from the top 100 articles edited for each user and the pairwise Jaccard similarity 
    \item \textbf{Talk and Interaction graphs}, measures communication between users on their respective user talk pages
    \item \textbf{Signed Graphs}, triad encoding and extracting the triad counts for each voter
\end{itemize}
\section{Models}
\subsection{Linear Combination of Graphs}
\begin{itemize}
    \item Discuss the various linear models considered for Graph Combinations
    \begin{itemize}
        \item Linear Regression
        \item Support Vector Classifier
        \item Extreme Gradient Boosting (XGBOOOST) 
    \end{itemize}
    \item Discuss how each graph contributes features and the problem is a linear classification problem
\end{itemize}
\subsection{Iterative Mode}
\begin{itemize}
    \item Discuss the motivation behind an iterative model versus a static prediction model
    \item Describe how balance is derived from the Agree Graph in a local signed network
    \item Discuss how the Agree graph is updated in terms of Balance
    \item Describe how status is derived from the Follows graph in a local signed network 
    \item Discuss how the Follows graph is updated after every election
    \item Describe how to make the predictions 
    \begin{itemize}
        \item Deterministic : just decide based on eigen value or agony as support or oppose
        \item Probabilistic : provide a probability for predicting a support vote
    \end{itemize}
\end{itemize}
\section{Evaluation}
\begin{itemize}
    \item Discuss the issues with the imbalance in the datasets
    \item Illustrate the issues with pure measures of accuracy
    \item Define Precision, Recall and Macro F1 score
    \item Discuss ROC AUC and Precision Recall curves for probability based predictions 
\end{itemize}

\chapter{Results and Discussion} 
\label{chp:results}
In this section we will present the results of the models and discuss their implications.
\section{Linear Combination of Graphs}
\begin{itemize}
    \item Present results for each linear classifier
    \item Discuss the different splits of the dataset to check for robustness and chronological consistency
    \item Show the feature importances and discuss their relevance 
    \item Compare the raw accuracy versus the macro f1 scores
    \item Highlight the difficulty of predicting negative votes
\end{itemize}
\section{Local Signed Network}
\begin{itemize}
    \item Present the Iterative Balance model results
    \item Discuss quality of predictions using evaluation metrics
    \item Mention the difference between deterministic and probabilistic prediction accuracies
    \item Explain the Iterative Status model results 
    \item Discuss the issues with local model of status and the potential reasons for lower score and quality
\end{itemize}
\section{Comparison}
\begin{itemize}
    \item Compare results from signed edge prediction and Iterative signed models
    \item Discuss Static Linear combination predictions versus Iterative signed predictions 
    \item Discuss the assumptions used in the models and limitations 
\end{itemize}

\chapter{Conclusions and Future Work}
\label{chp:conclusion}
\begin{itemize}
    \item Explain the quality of results with the election perspective
    \item Future work is to extend this to other election settings and investigate generality of this approach
    \item Possible future work in congressional voting data
    \item Can also tackle the other problem in information cascade theory of how to predict who is most likely to vote next 
    \item This can lead to a complete model of election dynamics and could incorporate elements of game theory and network inference 
\end{itemize}
