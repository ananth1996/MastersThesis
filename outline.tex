\chapter{Results and Discussion} 
\label{chp:results}
In this section we will present the results of the models and discuss their implications.
\section{Linear Combination of Graphs}
\begin{itemize}
    \item Present results for each linear classifier
    \item Discuss the different splits of the dataset to check for robustness and chronological consistency
    \item Show the feature importances and discuss their relevance 
    \item Compare the raw accuracy versus the macro f1 scores
    \item Highlight the difficulty of predicting negative votes
\end{itemize}
\section{Local Signed Network}
\begin{itemize}
    \item Present the Iterative Balance model results
    \item Discuss quality of predictions using evaluation metrics
    \item Mention the difference between deterministic and probabilistic prediction accuracies
    \item Explain the Iterative Status model results 
    \item Discuss the issues with local model of status and the potential reasons for lower score and quality
\end{itemize}
\section{Voting Order Results}
\begin{table}
    \centering
    \caption{Results for different vote orderings for the failed RfA}
    \label{tab:fail-rfa}
    \begin{tabular}{llccc}
        \toprule
        Model & Vote Order & ROC AUC & PR Positive  & Pr Negative \\ \midrule
        Baseline & - & 0.5 & 0.52 & 0.479 \\
        \midrule
        
        \multirow{3}{*}{\shortstack[l]{Iterative\\ Balance}} & 
        Normal &  0.543 & 0.74 & 0.454 \\
        \cmidrule{2-5}
        &Reversed & 0.74 & 0.868 & 0.572 \\
        \cmidrule{2-5}
        & Random & 0.68 & 0.812 & 0.575 \\
        \midrule

        \multirow{3}{*}{\shortstack[l]{Iterative\\ Status}} & 
        Normal & 0.962 & 0.977 & 0.909 \\
        \cmidrule{2-5}
        & Reversed & 0.930 & 0.965 & 0.806   \\
        \cmidrule{2-5}
        & Random & 0.924 & 0.957 & 0.818 \\
        \bottomrule
        \end{tabular}
\end{table}

\begin{table}
    \centering
    \caption{Results for different vote orderings for the successful RfA}
    \label{tab:pass-rfa}
    \begin{tabular}{llccc}
        \toprule
        Model & Vote Order & ROC AUC & PR Positive  & Pr Negative \\ \midrule
        Baseline & - & 0.5 & 0.905 & 0.095 \\
        \midrule
        
        \multirow{3}{*}{\shortstack[l]{Iterative\\ Balance}} & 
        Normal &  0.9175 & 0.991 & 0.385 \\
        \cmidrule{2-5}
        &Reversed & 0.720 & 0.972 & 0.142 \\
        \cmidrule{2-5}
        & Random & 0.894 & 0.989 & 0.431 \\
        \midrule
        
        \multirow{3}{*}{\shortstack[l]{Iterative\\ Status}} & 
        Normal & 0.846 & 0.981 & 0.293 \\
        \cmidrule{2-5}
        & Reversed & 0.895 & 0.99 & 0.29 \\
        \cmidrule{2-5}
        & Random & 0.931 & 0.992 & 0.451 \\
        \bottomrule
        \end{tabular}
\end{table}
\section{Comparison}
\begin{itemize}
    \item Compare results from signed edge prediction and Iterative signed models
    \item Discuss Static Linear combination predictions versus Iterative signed predictions 
    \item Discuss the assumptions used in the models and limitations 
\end{itemize}

\chapter{Conclusions and Future Work}
\label{chp:conclusion}
\begin{itemize}
    \item Explain the quality of results with the election perspective
    \item Future work is to extend this to other election settings and investigate generality of this approach
    \item Possible future work in congressional voting data
    \item Can also tackle the other problem in information cascade theory of how to predict who is most likely to vote next 
    \item This can lead to a complete model of election dynamics and could incorporate elements of game theory and network inference 
\end{itemize}
