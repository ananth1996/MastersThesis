\documentclass{beamer}
\usetheme{Madrid}

\AtBeginSection[]{
  \begin{frame}
  \vfill
  \centering
  \begin{beamercolorbox}[sep=8pt,center,shadow=true,rounded=true]{title}
    \usebeamerfont{title}\insertsectionhead\par%
  \end{beamercolorbox}
  \vfill
  \end{frame}
}


\usepackage{subcaption}
\usepackage{tikz}
\usetikzlibrary{positioning}
\usetikzlibrary{calc}
\usetikzlibrary{arrows}
\usetikzlibrary{arrows.meta}
\usetikzlibrary{decorations.pathmorphing,decorations.markings}
\usetikzlibrary{shapes}
\usetikzlibrary{patterns}
\usetikzlibrary{shadows,positioning}
\usepackage{amssymb}
\usepackage{amsmath}
\usepackage[ruled,linesnumbered]{algorithm2e}
\colorlet{colD}{red!40}
\colorlet{colIP}{cyan!40}
\colorlet{colV}{blue!40}
\colorlet{colBorder}{gray!70}


\title[Signed Vote Prediction]{Vote Prediction Models for Signed Social Networks}
\author{Ananth Mahadevan}
\institute{Aalto University}
\date{\today}

\begin{document}
\begin{frame}
    \titlepage
\end{frame}

\begin{frame}
    \frametitle{Overview}
    \tableofcontents
\end{frame}


\section{Voting and Signed Networks}

\begin{frame}
    \frametitle{Voting in Communities}
    \begin{itemize}
        \item Communities need to take collective action 
        \item Voting is a popular method 
        \item Members of the community vote on the agenda
        \item E.g.
        \begin{itemize}
            \item Politicians voting for bills in the parliament
            \item Wikipedia users voting for promoting administrators 
        \end{itemize}
        \item Understanding voting behaviour is beneficial
        \item Can propose agendas items which will be be successful 
        \item Identify ideological fault lines amongst members
    \end{itemize}
    
\end{frame}

\begin{frame}
    \frametitle{Votes as Signed Graphs}
    \begin{itemize}
        \item Votes are usually for or against an agenda
        \item Intuitively maps to positive and negative edges in signed graphs 
        \item More tools to analyse voting patterns, e.g.,
        \begin{itemize}
            \item Correlation clustering \cite{brito2020aBrazil,arinik2017signed}
            \item Balance and Status \cite{levorato2016brazilian,derr2018congressional}
        \end{itemize}
        \item Two main prediction tasks exist with regard to voting 
        \begin{enumerate}
            \item Predicting the Result
            \item Predicting an individual vote
        \end{enumerate}
        \item We focus on predicting votes
    \end{itemize}
    
\end{frame}

\begin{frame}
    \frametitle{Predicting Votes}
    Predicting a vote can be split into two phases
    \begin{enumerate}
        \item \textbf{Who} will vote next
        \begin{itemize}
            \item Same as link prediction task
            \item Is trivial when voting order is known, e.g., parliament roll calls
            \item Combinatorial if no known underlying process
        \end{itemize}
        \item \textbf{How} they will vote 
        \begin{itemize}
            \item Same as sign prediction task
            \item Triad features encode balance and status theory
            \item Train a supervised ML model using network and triad features \cite{leskovec2010predicting,leskovec2010signed}
        \end{itemize}
    \end{enumerate}
    We propose an \textit{unsupervised iterative model} to predict the sign of a vote using balance and status theory.

\end{frame}
\section{Local Signed Network}
\begin{frame}
    \frametitle{Terminology}
    \begin{itemize}
        \item Current voting sessions is a signed graph $S=(V_S,E_S,w_S)$
        \item It contains current voter $v$, candidate $c$ and prior voters $U$
        \item We also have a \textit{Relationship Graph} $R = (V_R,E_R,w_R)$
        \item It is created from the history of voting sessions $H$
        \item The \textit{Local Signed Network} is the intersection of these two graphs $LSN = S \cap R$
        \item Essentially the subgraph of the voter's neighbours in $R$ who have already voted in $S$
        \item Can use balance and status theory in the LSN to predict votes
    \end{itemize}
    

\end{frame}

\section{Balance Theory}

\subsection{Agreement Graph}
\begin{frame}
    \frametitle{Agreement Graph}

    

\end{frame}
\subsection{Model}
\begin{frame}
    \frametitle{Iterative Balance Model}
    \centering
    \scalebox{0.7}{\tikzset{
        position/.style args={#1:#2 from #3}{
            at=(#3.#1), anchor=#1+180, shift=(#1:#2)
        }
    }

\begin{figure}[!ht]
    \centering

\begin{tikzpicture}

    \begin{scope}[every node/.style={circle,thick,draw,minimum size=9mm}]
        \node (u1) at (0,0) {$u_1$};
        \node[position=-60:1.3cm from u1] (c) {$c$};
        \node[position=-120:1.3cm from u1] (v) {$v$};

        \node[right= 5cm of u1] (u2) {$u_2$};
        \node[left=1cm of u2] (u12) {$u_1$};
        \node[position=-90:1cm from u2] (c2) {$c$};
        \node[left=1cm of c2] (v2) {$v$};

        \path (c) -- node[below=3cm] (u3) {$u_3$} (v2); 
        \node[right= 1cm of u3] (u13) {$u_1$};
        \node[below=1cm of u3] (c3) {$c$};
        \node[right=1cm of c3] (u23) {$u_2$};
        \node[left=1cm of c3] (v3) {$v$};
    \end{scope}

    % positive edges
    \begin{scope}[
        relation/.style= {thick,draw,blue},
        vote/.style= {very thick,draw,red,densely dotted},
        predict/.style={thick,draw,dashed},
        every node/.style={fill=white,rectangle,inner sep=0}]
        % \draw (v4) -- (v1) -- (v3) -- (v2) -- (v1);
        \path 
        (u1) edge[vote] node[above right=1mm] {$-$} (c)  
        (v) edge[relation] node[above left=1mm] {$-$} (u1)
        (v) edge[predict] (c)
        
        (u12) edge[vote] node[above right=1mm] {$-$} (c2)  
        (u2) edge[vote] node[right=1mm] {$+$} (c2)
        (u2) edge[relation] node[above=1mm] {$-$} (u12)
        (v2) edge[relation] node[left=1mm] {$+$} (u12)
        (v2) edge[predict] (c2)

        (u13) edge[vote] node[above left=1mm] {$-$} (c3)  
        (u23) edge[vote] node[above=1mm] {$+$} (c3)
        (u3) edge[vote] node[left=1mm] {$-$} (c3)
        (u23) edge[relation] node[right=1mm] {$-$} (u13)
        (u3) edge[relation] node[above=1mm] {$+$} (u13)
        (v3) edge[relation] node[above left=1mm] {$-$} (u3)
        (v3) edge[relation,bend right=50] node[above=1mm] {$-$} (u23)
        (v3) edge[predict] (c3)

        
        ;
    \end{scope}
    \begin{scope}[
        every edge/.style= {},
        every node/.style={fill=white,rectangle,inner sep=0,text=black}]
    \path 
    (v) edge node[below=2cm] {$(a)~i=1$} (c)
    (v) edge node[below=1cm] {$\lambda_1^+ = 0, \lambda_1^-=1$} (c)
    
    (v2) edge node[below=2cm] {$(b)~i=2$} (c2)
    (v2) edge node[below=1cm] {$\lambda_1^+ = 0.58, \lambda_1^-=0$} (c2)
    (v3) edge node[below=3cm] {$(c)~i=3$} (u23)
    (v3) edge node[below=2cm] {$\lambda_1^+ = 0.55, \lambda_1^-=0.55$} (u23)
    ;
    \end{scope}
  \end{tikzpicture}
  


    \caption{Iterative Balance Local Singed Network model example. At every iteration $i$ the dotted red lines are prior votes, solid blue lines are relationships based on voting records and the dashed black edge $(v,c)$ is the edge whose sign is being predicted. $\lambda_1^+$ and $\lambda_1^-$ correspond to the smallest eigenvalue of the signed Laplacian $\overline{L}_i$ based on whether $w((v,c))=+1$ or $w((v,c))=-1$.     }
    \label{}
\end{figure}}

\end{frame}

\section{Status Theory}


\subsection{Follows Graph}

\begin{frame}
    \frametitle{Follows Graph}
    
    
    
\end{frame}

\subsection{Agony and Status}
\subsection{Model}
\begin{frame}
    \frametitle{Iterative Status Model}
        \centering
        \scalebox{0.5}{\tikzset{
        position/.style args={#1:#2 from #3}{
            at=(#3.#1), anchor=#1+180, shift=(#1:#2)
        }
    }

\begin{figure}[!ht]
    \centering

\begin{tikzpicture}

    \begin{scope}[every node/.style={circle,thick,draw,minimum size=9mm}]
        \node (u) at (0,0) {$u$};
        \node[position=-60:1.3cm from u] (c) {$c$};
        \node[position=-120:1.3cm from u] (v) {$v$};
        
        \node[position=-120:2cm from v] (u1)  {$u$};
        \node[position=-60:1.3cm from u1] (c1) {$c$};
        \node[position=-120:1.3cm from u1] (v1) {$v$};

        \node[position=-60:2cm from c] (u2)  {$u$};
        \node[position=-60:1.3cm from u2] (c2) {$c$};
        \node[position=-120:1.3cm from u2] (v2) {$v$};

        \node[position=-90:4cm from u1] (u3)  {$u$};
        \node[position=-60:1.3cm from u3] (c3) {$c$};
        \node[position=-120:1.3cm from u3] (v3) {$v$};

        \node[position=-90:4cm from u2] (u4)  {$u$};
        \node[position=-60:1.3cm from u4] (c4) {$c$};
        \node[position=-120:1.3cm from u4] (v4) {$v$};


    \end{scope}

    % positive edges
    \begin{scope}[
        relation/.style= {thick,draw,blue,->,>={Stealth[blue]}},
        vote/.style= {very thick,draw,red,densely dotted,->,>={Stealth[red]}},
        predict/.style={thick,draw,dashed,->,>={Stealth[black]}},
        assume/.style={thick,draw,->,>={Stealth[black]}},
        every node/.style={fill=white,rectangle,inner sep=0}]
        % \draw (v4) -- (v1) -- (v3) -- (v2) -- (v1);
        \path 
        (u) edge[vote] node[above right=1mm] {$-$} (c)  
        (v) edge[relation] node[above left=1mm] {$-$} (u)
        (v) edge[predict] (c)

        (u1) edge[vote] node[above right=1mm] {$-$} (c1)  
        (v1) edge[relation] node[above left=1mm] {$-$} (u1)
        (v1) edge[assume] node[above=1mm] {$+$} (c1)
        
        (u2) edge[vote] node[above right=1mm] {$-$} (c2)  
        (v2) edge[relation] node[above left=1mm] {$-$} (u2)
        (v2) edge[assume] node[above=1mm] {$-$} (c2)
        
        (c3) edge[vote] node[above right=1mm] {$+$} (u3)  
        (u3) edge[relation] node[above left=1mm] {$+$} (v3)
        (v3) edge[assume] node[above=1mm] {$+$} (c3)
        
        (c4) edge[vote] node[above right=1mm] {$+$} (u4)  
        (u4) edge[relation] node[above left=1mm] {$+$} (v4)
        (c4) edge[assume] node[above=1mm] {$+$} (v4)
        ;
    \end{scope}
    \begin{scope}[
        every edge/.style= {thick,draw,->},
        every node/.style={fill=white,rectangle,inner sep=0,text=black}
    ]
    \node[position=-90:4cm from u3] (a1) {$\alpha^+=3$};
    \node[position=-90:4cm from u4] (a2) {$\alpha^-=0$};
    \path     
    (v) edge[shorten >=0.5cm,shorten <=0.5cm] node[above left=0.1cm,text width=2cm,align=center] {Assume \\ positive vote} (u1)
    (c) edge[shorten >=0.5cm,shorten <=0.5cm] node[above right=0.1cm,text width=2cm,align=center] {Assume \\ Negative vote} (u2)

    ($(u1)!0.5!(u3)$) edge node[right=1mm] (convertL) {} ($(u1)!0.8!(u3)$)
    ($(u2)!0.5!(u4)$) edge node[left=1mm] (convertR) {} ($(u2)!0.8!(u4)$)
    
    (convertL) edge[draw=none] node[text width=3cm,align=center] {Transform\\ negative edges} (convertR)
    
    ($(u3)!0.6!(a1)$) edge node[right=1mm] (agonyL) {} ($(u3)!0.9!(a1)$)
    ($(u4)!0.6!(a2)$) edge node[left=1mm] (agonyR) {} ($(u4)!0.9!(a2)$)
    
    (agonyL) edge[draw=none] node[text width=3cm,align=center] {Compute\\ Agony} (agonyR)

    ;
    \end{scope}
  \end{tikzpicture}
  


    \caption{Example of LSN sign prediction using status theory.}
    \label{fig:lsn-status}
\end{figure}}
    

\end{frame}

\section{Iterative Prediction}

\begin{frame}
    \frametitle{Algorithm}
    \centering
    \scalebox{0.7}{
    \begin{algorithm}[H]
        \DontPrintSemicolon
        \KwIn{Candidate $c$, Relationship graph $R=(V_R,E_R,w_R)$, Order of voters in current session $O$ and true votes $w^*$  }
        \KwResult{Predictions for current session }
        $k \leftarrow |O|$\;
        $u \leftarrow O[1]$ \tcp*{First voter} 
        $V_S \leftarrow \{c,u\}$ \tcp*{candidate and first voter}
        $E_S \leftarrow \{(u,c)\}$ \tcp*{first vote}
        $w_S((u,c)) \leftarrow w^{*}((u,c))$ \tcp*{Assign true vote}
        Initialize session graph $S = \{V_S,E_S,w_S\}$\;
        $predictions \leftarrow \emptyset$ \;
        \For{$i \leftarrow 2$ \KwTo $k$}{
            $v \leftarrow O[i]$\; 
            $V_S \leftarrow V_S \cup \{v\}$ \;
            $LSN \leftarrow S \cap R$\;
            $p \leftarrow Predict(v,c,LSN)$\; 
            $predictions \leftarrow predictions \cup p$\;
            $E_S \leftarrow E_S \cup \{(v,c)\}$\;
            $w_S((v,c)) \leftarrow w^{*}((v,c))$ \tcp*{Assign true vote}
        }   
        $Update(R,S)$ \tcp*{Update Relationship graph}
        \Return $predictions$\;
    \end{algorithm}
    }
\end{frame}
\section{Results}


\bibliography{../sources}
\bibliographystyle{apalike}

\end{document}