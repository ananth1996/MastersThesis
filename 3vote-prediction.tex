\chapter{Vote Prediction}
\label{chp:vote-prediction}
In this chapter, we cover the main motivation behind predicting the vote of an individual voter and present the methods that can be used to solve this task.
First, we discuss the contrast in perspectives that is present when predicting the result compared to predicting a vote in Section~\ref{sec:result-vs-vote}.
Next, in Section~\ref{sec:voting-signed-networks}, we explain how the problem of vote prediction is intrinsically linked to the tasks of link and sign prediction in a signed network.
In Section~\ref{sec:linear-combination-theory}, we describe a supervised machine learning framework that uses features from graphs for voting and non-voting data.
Lastly, we present our novel approach of constructing a local signed graph of the current voter and using balance and status theory to predict the vote in Section~\ref{sec:local-signed-network-theory}. 


\section{Result versus Vote Prediction}
\label{sec:result-vs-vote}
In this thesis, we are interested in the voting behaviour for a collective action.
In such cases, members of a community come together as \textit{voters} to decide on a particular \textit{candidate} item during a \textit{session}.
In a parliament  the voters are the elected members of the parliament and the candidate is usually a bill or a policy matter.
When it is promotion within a political party or an online community such as Wikipedia, the members vote on a candidate who has been nominated for the position.
In all these cases, we have two levels of decisions being made.
The first is the individual decision that each voter makes with regard to the candidate.
The second is the final decision that the group arrives at after everyone has voted.
We refer to task of predicting the former as \textit{vote prediction} and the latter as \textit{result prediction}. 

Result prediction provides a macro level perspective of the incentives of a community.
We can create models based on the characteristics of a candidate to predict the result of a collective action.
This leads to an understanding on a communal level, of what characteristics are preferred and if there are voting blocks formed within the members based on ideological differences.
This translates to practical examples such as party level dynamics in a parliament, the topic of a bill or the credentials of a nominee \cite{burke2008mopping,yano2012textual,yogatama-etal-2011-predicting}. 

On the other hand, when we focus on the vote prediction problem, we get a deeper understanding of the dynamics between voters and the candidate.
In fields such as game theory, this is well studied using frameworks such as \textit{strategic voting models} and \textit{momentum} \cite{meir2020strategic,zou2015strategicDoodle,ali2006a,banerjee1992simple,tal2015a}.
These models are more theoretical and are studied under synthetic conditions.
Nevertheless, they still provide a foundation on which we can construct practical models that can utilize additional features.
One such popular approach is using textual information from bills, speeches and legislature to predict votes of politicians in parliament \cite{budhwar2018predicting,gerrish2011predicting}.
The next important step is to represent the voting data as networks and leveraging network features to understand and predict voter behaviour.


\section{Voting and Signed networks}
\label{sec:voting-signed-networks}
Votes by nature express a positive or negative relationship between a \textit{voter} and a \textit{candidate}.
Therefore, \textit{signed graphs} provide an intuitive way to structurally represent the voting pattern of members in a community.
These signed voting networks can be used to develop models to solve the task of vote prediction and analyse voter behaviour.

Correlation clustering and community detection of signed voting graphs can discover trends and vote blocks in communities \cite{brito2020aBrazil,arinik2017signed}.
Analysing the networks using social theories of balance and status provides knowledge of voter behaviour and features for prediction methods \cite{levorato2016brazilian,derr2018congressional}.
The vote prediction task can be broken down into two subtasks which are analogous to link and sign prediction in signed networks. 

The first subtask is to predict \textit{who} will vote next given a candidate $c$ and a set of previous votes.
This subtask is similar to link prediction in a signed network, where we aim to predict possible future edges of the type $(v,c)$.
The complexity of this subtask depends on the format of voting that takes place.
For instance, if there is a known voting order, such as roll calls in parliamentary proceedings or explicit timestamps, then the subtask is trivially solved.
On the other hand, if the voting occurs simultaneously, the subtask can be reduced to predicting the possible subset of members who will vote in a given session.
However, when the voting is iterative and there is no known underlying process of who votes next, then a separate model might be required to infer the voting sequence.
This particular case can be combinatorially hard, as we would need to find the correct ordering of votes in a session.

The second subtask is to predict \textit{how} a voter $v$ will vote for a candidate $c$ given the previous votes in the session.
This task translates into predicting the sign of an edge $(v,c)$, in the current session.
We call this subtask the \textit{independent vote prediction} problem.
It is independent as we are only interested in the decision of the voter $v$, assuming that we have complete prior knowledge of how the previous votes have been cast.
Then, this problem can be framed as a supervised learning task.
It uses features of the interaction between the current voter $v$, previous voters $U$ and the candidate $c$ to predict the sign of the edge $(v,c)$.
We can utilize the theories of balance and status in signed networks to create models similar to those by Karimi et al.\ \cite{karimi2019multicongress} and Jankowski-Lorek et al.\ \cite{jankowski-lorek2013MBSN} to predict voter behaviour.

\iffalse
\todo[inline]{Are the IC paras neccessary?}
Information cascades are widely studied in epidemiology where we model the spread of a contagion through infections of neighbours and the halting using a rate of recovery.
Collective voting in a network can be similarly modelled as a sequence of votes spreading through a network and naturally ending when a final decision has been taken.
Using an \textit{independent cascade} model we can separate the spread of information in a network into two logical problems.

The first, is a which node will be affected next.
This is equivalent to knowing which node will get infected in a contagion model or which person will be next to vote in a voter model.
In contagions spread this is probabilistic hence can be easily modelled.
This is different for voting models as some cases might have a fixed voting order while others may not.
In the cases we have a known voting order such as the roll call for voting in parliament or timestamps in promotions of Wikipedia administrators, the problem is intrinsically solved.
We have mentioned other models above, that can handle cases where there is either no known order or voting happens in a non-sequential manner. 

The second problem, is how will the node will react.
In the context of a contagion model, this translates to whether the now infected node becomes sick or not.
This is usually a known feature of a given contagion and hence is not critical to a contagion model.
In a voter mode, this problem is determining how an individual voter will cast their vote given the information of who have already cast their vote.
We call this problem the \textit{independent vote prediction} problem as it considers each voter in the chain as an independent actor and only aims to predict how that particular individual would vote.
In solving this problem we can gain insights into the motivations and behaviour of voters. 

The problem of independent vote prediction is studied in Karimi et al.\ \cite{karimi2019multicongress} and Jankowski-Lorek et al.\cite{jankowski-lorek2013MBSN}.
They propose models that incorporate network information, textual data, signed graph features to predict the voting behaviour.
Although their models achieve a high predictive accuracy in their respective tasks, they are quite tasks specific and cannot easily be extended to other settings.
In contrast, we focus on using general signed network features of balance and status to create a model to predict the votes. 

\fi

\section{Linear Combination of Graphs}
\label{sec:linear-combination-theory}

In this section we explain how the approaches outlined in Section~\ref{sec:link-prediction} can be applied to solve the \textit{independent vote prediction} problem.
As discussed previously, the \textit{edge sign prediction} task in signed network is analogous to vote prediction.
The models proposed by Leskovec et al.\ \cite{leskovec2010predicting} can be used to predict the sign of the edge $(v,c)$.
However, voting in a community takes place across many sessions in a chronological manner.
Therefore, we must partition the training and testing datasets to avoid data leakage.
We propose the following framework to gather features using a linear combination of the voting history and several auxiliary graphs.

We denote the directed signed graph for the current voting session as $S=(V_S,E_S,w_S)$.
The current voter in consideration is denoted by $v$ and the candidate of the session is $c$.
In this thesis, for all the models we assume that each session has a unique candidate.
The votes that have been cast prior to the current voter exists as edges $(u,c)$ in the session $S$ and the set of prior voters is denoted as $U =\{u \mid (u,c) \in E_S\}$.
The history $H=(V_H,E_H,w_H)$ is defined as the directed signed graph containing the votes from sessions chronologically prior to $S$.

We define a set of auxiliary graphs $A = \{G_1,G_2,\dots,G_l\}$ based on external non-voting data.
These auxiliary graphs can be either directed or undirected, weighted or unweighted, signed or unsigned.
This is similar to the relational layer in \textit{Multidimensional Social Networks} (MSN) described by Kazienko et al.\ \cite{kazienko2011multidimensional} and Jankowski-Lorek et al.\ \cite{jankowski-lorek2013MBSN}.
However, the auxiliary graphs capture different relations between a \textit{subset} of the voting members that need not overlap.
These relations will be used to generate additional features for the vote prediction task. 

Algorithm~\ref{alg:auxiliary-feature} describes how to generate a feature vector $\mathbf{a}$ from the auxiliary graphs $A$.
The algorithm finds the intersection of the prior voters $U$ and the neighbourhood of the current voter $v$ in the graph $G_i$ which we call the \textit{voting neighbourhood}.
Then the feature $\mathbf{a}_i$ is computed as the weighted sum of the voting neighbourhood plus the edge weight to the candidate $(v,c)$ in the auxiliary graph $G_{i}$.
Figure~\ref{fig:linear-combination of graphs} provides an example with three auxiliary graphs and two prior voters $u_1$ and $u_2$.
The dashed red edges are the votes cast in the current session $S$ by the prior voters.
We see, in the example, $G_1$ is a directed graph, $G_2$ is an undirected graph and $G_3$ is a signed directed graph.
The current voter $v$ has different relations to his voting neighbourhood in each auxiliary graph and therefore each feature is a different combination of edge weights from those graphs. 

In addition to the auxiliary feature vector $\mathbf{a}$, we can also create triad features based on the historical voting graph $H$.
Similar to Leskovec et al.\ \cite{leskovec2010predicting} and Karimi et al.\ \cite{karimi2019multicongress} we can form a set of unique triads $T$.
Then, for each node $u$ in the common neighbourhood of $N_{vc}$ we can count the triad formed from the three vertices.
Algorithm~\ref{alg:triad-feature} describes this procedure. 

We now create a feature matrix $\mathbf{X}$ for all the sessions in a given dataset.
In the feature matrix, each row is the concatenation of the auxiliary feature vector and the triadic feature vector $\mathbf{x} = [\mathbf{a},\mathbf{t}]$ .
The target vector $\mathbf{y}$ is the vector of true votes.
Now, we train a linear machine learning model and use it to predict the votes in a test dataset.

\begin{algorithm}[htp]
    \DontPrintSemicolon
    \caption{Auxiliary feature vector for voter $v$ }
    \label{alg:auxiliary-feature}
    \KwIn{Voter $v$, Candidate $c$, Set of auxiliary graphs $A$, Current voting session $S$ and Prior voters $U$ }
    \KwResult{Auxiliary Feature vector $\mathbf{a}$ }
    Initialize $\mathbf{a}$ of length $|A|$\;
    \ForEach{$G_i$ in $A$}{
     
      $Z = N_i(v) \cap U$ \tcp*{neighbours in $G_i$ who have voted} \label{alg:aux:voting-neighbourhood}
      $\mathbf{a}_i \leftarrow 0$\;
      \ForEach{$z$ in $Z$}{
          \tcc{vote in current session multiplied by the edge weight in $G_i$}
          $\mathbf{a}_i$ += $w_S((z,c)) \cdot w_i((v,z))$
      }
      $\mathbf{a}_i += w_i((v,c))$ \tcp*{Add edge weight to candidate}
    }
    \Return $\mathbf{a}$
\end{algorithm}

\begin{figure}[!ht]
    \centering
    \tikzset{
    position/.style args={#1:#2 from #3}{
        at=(#3.#1), anchor=#1+180, shift=(#1:#2)
    }
}

\begin{tikzpicture}

    \begin{scope}[every node/.style={circle,thick,draw,minimum size=7mm}]
        \node (v) at (0,0) {$v$};
        \node[right= 2cm of v] (c) {$c$};
        \node[position=50:1cm from v] (u1) {$u_1$};
        \node[position=-50:1cm from v] (u2) {$u_2$};
        
        \node[right = 4cm of v] (v1)  {$v$};
        \node[right= 2cm of v1] (c1) {$c$};
        \node[position=50:1cm from v1] (u11) {$u_1$};
        \node[position=-50:1cm from v1] (u21) {$u_2$};
        
        \node[right = 4cm of v1] (v2)  {$v$};
        \node[right= 2cm of v2] (c2) {$c$};
        \node[position=50:1cm from v2] (u12) {$u_1$};
        \node[position=-50:1cm from v2] (u22) {$u_2$};
        
    \end{scope}

    % Session vote edges
    \begin{scope}[
        >={Stealth[red]},
        every edge/.style={thick,draw,red,->,dashed},
        every node/.style={fill=white,circle,inner sep=0pt}]
        \path 
        (u1) edge node[above right=1mm] {$+$} (c)
        (u2) edge node[below right=1mm] {$-$} (c)

        (u11) edge node[above right=1mm] {$+$} (c1)
        (u21) edge node[below right=1mm] {$-$} (c1)

        (u12) edge node[above right=1mm] {$+$} (c2)
        (u22) edge node[below right=1mm] {$-$} (c2)
        ;
    \end{scope}
    \path (v) edge[white] node[below=2.2cm,text=black] {$G_1$} (c);
    \path (v) edge[white] node[below=2.8cm,text=black] {$\mathbf{x}_1=w_a+w_b$} (c);
    
    \path (v1) edge[white] node[below=2.2cm,text=black] {$G_2$} (c1);
    \path (v1) edge[white] node[below=2.8cm,text=black] {$\mathbf{x}_2=w_c-w_b$} (c1);
    
    \path (v2) edge[white] node[below=2.2cm,text=black] {$G_3$} (c2);
    \path (v2) edge[white] node[below=2.8cm,text=black] {$\mathbf{x}_3=-(w_e+w_f+w_g)$} (c2);
    % Auxiliary graph edges
    \begin{scope}[
        >={Stealth[blue]},
        every edge/.style={thick,draw,blue,->},
        every node/.style={fill=white,circle,inner sep=0pt}]
        \path 
        (v) edge node[above left=1mm] {$w_a$} (u1)
        (v) edge node[above=1mm] {$w_b$} (c)

        (v1) edge[-] node[above left=1mm] {$w_c$} (u11)
        (v1) edge[-] node[below left=1mm] {$w_d$} (u21)

        (v2) edge node[above left=1mm] {$-w_f$} (u12)
        (v2) edge node[below left=1mm] {$w_g$} (u22)
        (v2) edge node[above] {$-w_e$} (c2) 
        ;
    \end{scope}

  \end{tikzpicture}
  
    \caption{Example auxiliary features for $v$ from combination of three graphs and two prior voters $u_1$ and $u_2$. Dashed red lines are prior votes in the session. Solid blue lines are edge weights in auxiliary graph. $\mathbf{a}_i$ is feature for voter $v$ from auxiliary graph $G_i$}
    \label{fig:linear-combination of graphs}
\end{figure}

\begin{algorithm}[H]
    \DontPrintSemicolon
    \caption{Triad feature vector for voter $v$ }
    \label{alg:triad-feature}
    \KwIn{Voter $v$, Candidate $c$, Set of unique triads $T$, Voting history graph $H$ }
    \KwResult{Triad Feature vector $\mathbf{t}$ }
    $k \leftarrow |T|$\;
    Initialize counters $cnt_1,\dots,cnt_k$ to 0\;
    $N_{vc} = N_H(v) \cap N_H(c)$ \tcp*{common neighbours in $H$}
    \ForEach{$u$ in $N_{vc}$}{
        Let $\triangle$ be the triad formed by vertices $\{v,u,c\}$ \;
        Classify $\triangle$ as the $j$th triad in $T$ \;
        Increment counter $cnt_j$
    }
    $\mathbf{t} \leftarrow [cnt_1,\dots,cnt_k]$ \;
    \Return $\mathbf{t}$
\end{algorithm}


\section{Local Signed Network}


\label{sec:local-signed-network-theory}
In this section, we present an unsupervised method that can be used iteratively to predict the votes in a session.
This method builds on top the concept of a \textit{voting neighbourhood}, introduced in the previous section and relies solely on the social theories of balance and status in signed networks to predict a vote.
However, unlike the triadic features used in the supervised method, we propose to utilize generic measures of a network's adherence to the theories of balance or status.
Then, we predict the vote that preserves these measures the best. 

As defined in the previous section, the directed signed graph of the current session is $S = (V_S,E_S,w_S)$ where, $v$ is the voter, $c$ is the candidate and $U$ is the set of prior voters in the session. $H=(V_H,E_H,w_H)$ is the directed signed graph that contains the historical voting records prior to the current session.

The first step is to construct a signed \textit{relationship graph} $R=(V_R,E_R,w_R)$ from the historical voting graph $H$.
The edges of this graph capture simple signed relationships between the voters such as \textit{agreement} or \textit{concurrence}.
The relationship graph can also be constructed uniquely based on the details of the domain in which the voting occurs.
Based on whether we use balance or status theory to predict the vote, the relationship graph is either unweighted or weighted respectively.

We now define the \textit{Local Signed Network} (LSN) of a voter $v$ as $LSN = S \localintersect R$.
Where, $\localintersect$ is the \textit{local intersection} of two graphs.
Given two graph $G_1=(V_1,E_1)$ and $G_2=(V_2,E_2)$, the local intersection is defined as seen in Equation~\eqref{eqn:localintersect}.
\begin{equation}
    \label{eqn:localintersect}
    G_1 \localintersect G_2 = (V_1 \cap V_2,E_1\cup E_2).
\end{equation}
Therefore, the LSN is the subgraph induced by the neighbourhood of $v$ in $R$ who have voted in the session $S$.
We assume that candidate $c$ is present in the relationship graph so that we get all the prior vote edges $(u_j,c)$ in the LSN.
The \textit{voting neighbourhood} that we described in the previous section, has the same set of vertices as that of the LSN, i.e., $V_{LSN}=V_S \cap V_R$.
There are three main types of edges present in the LSN.
The first are the prior vote edges $(u_j,c)$, from the prior voters to the candidate.
The second are the relationship edges $(v,u_j)$, from the voter to the prior voters.
The third is the relationships between the prior voters $(u_j,u_k)$.
All these three types of edges can be seen in the undirected LSN shown in Figure~\ref{fig:lsn-balance}c.
We will now explain how to predict the edge $(v,c)$ in the LSN using balance and status theories.  

\subsection{Prediction Based on Balance Theory}
\label{subsec:prediction-based-balance}
As described in Section~\ref{sec:balance-theory}, balance theory is applied to undirected signed graphs.
Therefore, we construct an undirected signed relationship graph $R$ using the voting history.
The edge set $E_{R}$, represents the signed relations between nodes and should be symmetric in nature.
For example, the probability of agreements or disagreement between a pair of nodes $u$ and $v$.
Now, we create the LSN from the intersection of the session graph $S$ and relationship graph $R$.
To keep the LSN an undirected graph, we ignore the direction of the voting edges $(u_j,c)$. 

Now, we turn to the task of predicting the sign of the edge $(v,c)$ in the LSN.
The voter $v$ can cast either a positive or negative vote for the candidate $c$.
This gives us two possibilities $w_{LSN}((v,c))=+1$ or $w_{LSN}((v,c))=-1$.
We state that the voter aims to maintain the balance in the LSN.
Therefore, we can predict the vote that when added as the edge $(v,c)$ to the LSN results in a more balanced network.
This indicates that we require a measure of the \textit{imbalance} of a network.
If $\overline{L}_{LSN}$ is the signed Laplacian of the LSN with eigenvalues $\lambda_1\leq\lambda_2\leq\dots\lambda_n$, then Li et al.\ \cite{li2016note} show that the smallest eigenvalue $\lambda_1$ is a measure of the imbalance of the LSN.
Therefore, the larger $\lambda_1$ is, the more imbalanced the network is and if $\lambda_1=0$, then the network is perfectly balanced. 
 
Combining these concepts we have Algorithm~\ref{alg:balance-pred}.
When given a LSN and an edge $(v,c)$ to predict, we first assume the vote is positive, add it to the LSN and compute the smallest eigenvalue denoted by $\lambda_1^{+}$.
Next, we assume the vote is negative and add the edge to the LSN and compute the corresponding smallest eigenvalue denoted by $\lambda_1^-$.

Now, if $\lambda_1^+ < \lambda_1^-$, then we can predict that the vote is positive, as the resulting LSN is more balanced.
Similarly, if $\lambda_1^-<\lambda_1^+$, we predict a negative vote, as it results in a more balanced network.
This deterministic rule does not capture the gap between $\lambda_1^+$ and $\lambda_1^-$.
Therefore, the ratio $r=\frac{\lambda_1^-}{\lambda_1^+}$ can be used to express the confidence in predicting the vote is positive.
As $\lambda_1^+$ approaches $0$ (or $\lambda_1^-$ increases), $r$ approaches $\infty$ and when $\lambda_1^-$ approaches $0$ (or $\lambda_1^+$ increases), $r$ approaches $0$.
Also, when $\lambda_1^+=\lambda_1^-$ then $r=1$, which indicates that we are equivocal between the vote being positive and negative.
We convert the ratio $r$ into a measure of the probability that the given edge $(v,c)$ is positive by defining $p=1/(1+e^{(1-r)})$.
Therefore, the output of the model is the probability that a vote is positive.

Figure~\ref{fig:lsn-balance} shows an example of how we can iteratively predict votes using balance theory.
The current voter at every iteration $i$ is $v=u_{i+1}$.
We start in the first iteration $i=1$ with one prior voter $u_1$ who has voted negatively for the candidate $c$ seen by the dotted red line in Figure~\ref{fig:lsn-balance}a.
The current voter $v$ has a negative relationship with $u_1$ indicated by the solid blue line.
Now, in this triad we know "$v$ disagrees with $u_1$" and "$u_1$ disapproves of $c$".
Using the intuition provided by balance theory, we can predict that the edge $(v,c)$ is positive as it results in the triad being balanced.
This result can be verified empirically by observing the values of smallest eigenvalue.
We see that $\lambda_1^+=0$ and $\lambda_1^-=1$ and therefore, the positive vote probability is $p=1$.
Now, in the next iteration, $v$ becomes the prior voter $u_2$ and we add the edge $(u_2,c)$ with the true vote (in this example we assume it was the same as the prediction made) and we get the next voter as the new $v$.

In the second iteration $i=2$, the new voter $v$ has a positive relation with the prior voter $u_1$ as seen in Figure~\ref{fig:lsn-balance}b.
By preserving the relation between $u_1$ and $u_2$ from the previous iteration, we have larger cycles in the current LSN.
Similar to the previous iteration, we observe the smallest eigenvalues of the LSN.
We conclude that a negative vote leads to more balanced network and positive vote probability is now $p=0$.
Now, in the third iteration $i=3$, the smallest eigenvalue of the LSN in both cases are equal.
Therefore, we are equivocal in the vote being positive or negative.
Hence, the positive vote probability is $p=0.5$.

\begin{figure}[!ht]
    \centering
    \tikzset{
        position/.style args={#1:#2 from #3}{
            at=(#3.#1), anchor=#1+180, shift=(#1:#2)
        }
    }

\begin{figure}[!ht]
    \centering

\begin{tikzpicture}

    \begin{scope}[every node/.style={circle,thick,draw,minimum size=9mm}]
        \node (u1) at (0,0) {$u_1$};
        \node[position=-60:1.3cm from u1] (c) {$c$};
        \node[position=-120:1.3cm from u1] (v) {$v$};

        \node[right= 5cm of u1] (u2) {$u_2$};
        \node[left=1cm of u2] (u12) {$u_1$};
        \node[position=-90:1cm from u2] (c2) {$c$};
        \node[left=1cm of c2] (v2) {$v$};

        \path (c) -- node[below=3cm] (u3) {$u_3$} (v2); 
        \node[right= 1cm of u3] (u13) {$u_1$};
        \node[below=1cm of u3] (c3) {$c$};
        \node[right=1cm of c3] (u23) {$u_2$};
        \node[left=1cm of c3] (v3) {$v$};
    \end{scope}

    % positive edges
    \begin{scope}[
        relation/.style= {thick,draw,blue},
        vote/.style= {very thick,draw,red,densely dotted},
        predict/.style={thick,draw,dashed},
        every node/.style={fill=white,rectangle,inner sep=0}]
        % \draw (v4) -- (v1) -- (v3) -- (v2) -- (v1);
        \path 
        (u1) edge[vote] node[above right=1mm] {$-$} (c)  
        (v) edge[relation] node[above left=1mm] {$-$} (u1)
        (v) edge[predict] (c)
        
        (u12) edge[vote] node[above right=1mm] {$-$} (c2)  
        (u2) edge[vote] node[right=1mm] {$+$} (c2)
        (u2) edge[relation] node[above=1mm] {$-$} (u12)
        (v2) edge[relation] node[left=1mm] {$+$} (u12)
        (v2) edge[predict] (c2)

        (u13) edge[vote] node[above left=1mm] {$-$} (c3)  
        (u23) edge[vote] node[above=1mm] {$+$} (c3)
        (u3) edge[vote] node[left=1mm] {$-$} (c3)
        (u23) edge[relation] node[right=1mm] {$-$} (u13)
        (u3) edge[relation] node[above=1mm] {$+$} (u13)
        (v3) edge[relation] node[above left=1mm] {$-$} (u3)
        (v3) edge[relation,bend right=50] node[above=1mm] {$-$} (u23)
        (v3) edge[predict] (c3)

        
        ;
    \end{scope}
    \begin{scope}[
        every edge/.style= {},
        every node/.style={fill=white,rectangle,inner sep=0,text=black}]
    \path 
    (v) edge node[below=2cm] {$(a)~i=1$} (c)
    (v) edge node[below=1cm] {$\lambda_1^+ = 0, \lambda_1^-=1$} (c)
    
    (v2) edge node[below=2cm] {$(b)~i=2$} (c2)
    (v2) edge node[below=1cm] {$\lambda_1^+ = 0.58, \lambda_1^-=0$} (c2)
    (v3) edge node[below=3cm] {$(c)~i=3$} (u23)
    (v3) edge node[below=2cm] {$\lambda_1^+ = 0.55, \lambda_1^-=0.55$} (u23)
    ;
    \end{scope}
  \end{tikzpicture}
  


    \caption{Local Singed Network prediction based on balance theory. At every iteration $i$ the dotted red lines are prior votes, solid blue lines are relationships based on voting records and the dashed black edge $(v,c)$ is the edge whose sign is being predicted. $\lambda_1^+$ and $\lambda_1^-$ correspond to the smallest eigenvalue of the signed Laplacian $\overline{L}_i$ based on whether $w((v,c))=+1$ or $w((v,c))=-1$.     }
    \label{}
\end{figure}
    \caption{Local Signed Network prediction based on balance theory. At every iteration $i$ the dotted red lines are prior votes, solid blue lines are relationships based on voting records and the dashed black edge $(v,c)$ is the edge whose sign is being predicted. $\lambda_1^+$ and $\lambda_1^-$ correspond to the smallest eigenvalue of the signed Laplacian $\overline{L}_i$ based on whether $w((v,c))=+1$ or $w((v,c))=-1$.     }
    \label{fig:lsn-balance}
\end{figure}

\begin{algorithm}[htp]
    \DontPrintSemicolon
    \caption{Predict positive vote probability using balance theory}
    \label{alg:balance-pred}
    \KwIn{Voter $v$, Candidate $c$, Local Signed Network $LSN$}
    \KwResult{Probablity of edge $(v,c)$ being positive}
    $w_{LSN}((v,c)) \leftarrow +1$ \tcp*{Assume positive vote}
    Compute signed Laplacian $\overline{L}_+$\;
    $\lambda_1^+ \leftarrow $ smallest eigenvalue of $\overline{L}_+$\;
    $w_{LSN}((v,c)) \leftarrow -1$ \tcp*{Assume negative vote} 
    Compute signed Laplacian $\overline{L}_-$\;
    $\lambda_{1}^- \leftarrow$ smallest eigenvalue of $\overline{L}_-$\;
    $w_{LSN}((v,c)) \leftarrow 0$ \tcp*{Reset edge weight}
    $r \leftarrow {\lambda_{1}^-}/{\lambda_{1}^{+}}$\;
    $p \leftarrow 1/(1+e^{(1-r)})$\;
    \Return p
\end{algorithm}

\subsection{Prediction Based on Status Theory }
\label{subsec:prediction-based-status}
We mentioned in Section~\ref{sec:status-theory} that status theory is defined for directed signed networks.
The relationship graph $R$ should, therefore, be constructed from the history $H$ as a directed signed network.
The directed edges $(u,v)$ should denote asymmetric relation between the nodes.
For example, the ratio of \textit{concurrence} which is a measure of times that $u$ has voted after $v$ in a session and agreed.
The LSN created from the intersection of the session graph $S$ and the relationship graph $R$ is also a directed signed graph.

Similar to predicting the vote using balance theory, the vote can either be positive or negative.
Now, we state that the voter aims to maintain the status in the resulting LSN.
Therefore, we predict the vote $(v,c)$ that when added to the LSN best preserves the status of the network.
Therefore, we require a measure of how much a given network adheres to the theory of status.
However, to the best of our knowledge, there are no existing methods to quantify and measure the extent that a network conforms to status theory.
In this thesis, we present a quantitative definition of status theory in a network and also present a novel method of using the agony of a directed network to measure the compliance with status theory.

As we discussed in Section~\ref{sec:status-theory}, signed edges can be interpreted as recognition of relative status.
This means that a positive edge $u \plusrightarrow v$ indicates that $v$ has a higher status than $u$, and a negative edge $u \minusrightarrow v$ indicates that $v$ has lower status than $u$.
Let us assume that there exists an \textit{implicit} status function $\sigma: V \rightarrow \mathbb{N}$ that maps each node in the network to a quantity that correlates with status.
Therefore, the edges $u\plusrightarrow v$ and $u \minusrightarrow v$ indicate $\sigma(u)<\sigma(v)$ and $\sigma(u)>\sigma(v)$ respectively.
We can use this relative ranking of status to predict the missing sign of an edge in a triad.
Leskovec et al.\ \cite{leskovec2010signed} provide a way to compute this status function $\sigma$ from the node degrees as shown in Section~\ref{sec:status-theory}.
However, this measure is still defined locally and can be used only to predict the sign of an edge.
We require a framework to define violations to status theory and measure how much a network on a whole complies with status theory.
We define that an edge $e=(u,v)$ \textit{violates} status theory if $w(e)\sigma(u)\not\leq w(e)\sigma(v)$.
Now, we can define when a network is perfectly complaint with status.

 \theoremstyle{definition}
 \begin{definition}{Perfect Status Compliance}
     A directed signed network $G=(V,E,w)$ with an implicit status function $\sigma: V \rightarrow \mathbb{N}$ is said to be perfectly status complaint if $\forall e=(u,v) \in E$, $\sgn(w(e))\sigma(u) \leq \sgn(w(e))\sigma(v)$.
 \end{definition}

Therefore, the number of edges that are in violation to status theory can be a rudimentary measure of the status compliance of a given network.
However, in the case of most signed directed networks, we do not possess the implicit status function $\sigma$.
Hence, even computing the number of edges that violate status theory is not possible.
However, if we can infer the status function $\sigma$ from the structure of the signed network then we can measure the status compliance of that network. 

We now use the concept of \textit{agony} described in Section~\ref{sec:hierarchy} to find the optimal hierarchy of a given directed network.
However, agony and hierarchy were primarily defined for unsigned networks.
We can easily remedy this by using the fact that a negative edge $u \minusrightarrow v$ can be transformed into a positive edge $u \plusleftarrow v$.
If $G$ is a signed network then we denote the unsigned directed network obtained from the transformation described as $G^\prime$.

The notion of hierarchy is strongly related to status theory for signed networks.
For instance, if $G^\prime$ is a DAG, then there are no cycles and we can find a status function $\sigma$, such that there are no edges that violate status theory.
Therefore, let us consider that the rank function $r:V\rightarrow \mathbb{N}$ used to define agony is the status function $\sigma$ for the signed network.
Then, the agony of an edge in $G^\prime$ is a measure of the violation of that edge with respect to status theory.

The agony of an edge $(u,v)$ in $G^\prime$ given the status function $\sigma$ is defined as $\max(\sigma(u)-\sigma(v)+1,0)$.
Therefore, agony is a quantification of the status violation of an edge.
The agony of the network with respect to a status function $\sigma$ is defined as the sum of the agony of each edge in the network denoted by $A(G^\prime,\sigma)$.
The agony of the network $A(G^\prime)= \min_\sigma(A(G^\prime,\sigma))$. is the smallest agony over all possible ranking of the nodes.
In this way, the agony of the network is a more generalized measure of status compliance than just counting the number of violating edges.
Therefore, we can use one of the algorithms mentioned in Section~\ref{sec:hierarchy} to compute agony of $G^\prime$.
We call this value as the agony of the original signed network $G$ and denote $\alpha = \alpha(G) = A(G^\prime)$.
The complete process is outlined in Algorithm~\ref{alg:signed-agony}.
Now, the agony of a signed network $G$ is a measure of how far $G$ is from being perfectly status complaint.
Theorem~\ref{the:agony-status} shows that when $\alpha=0$ the network is perfectly status complaint. 

\begin{theorem}
\label{the:agony-status}
Let $G=(V,E,w)$ be a directed signed graph.
Then $\alpha(G)=0$ if and only if $G$ is perfectly status complaint.
\end{theorem}
\begin{proof}
    The transformed unsigned directed network is $G^\prime=(V^\prime,E^\prime,w^\prime)$. The agony of a the directed network $G^\prime$ is 0 when the network has perfect hierarchy \cite{gupte2011finding}. Therefore, there exists a status function $\sigma$ such that the agony of all edges $G^\prime$, i.e., $\sigma(u) \leq \sigma(v), \forall (u,v)\in E^\prime$. Therefore, there are no edges in $G$ that violate status theory and $G$ is perfectly status complaint.
    Similarly, when $G$ is perfectly status compliant, we can find a status function $\sigma:V\rightarrow \mathbb{N}$, such that the agony of each edge in $G^\prime$ is $0$.
    Therefore, the agony of the entire graph $G^\prime$ is $0$, and correspondingly $\alpha(G)=0$.
    Hence proved.
\end{proof}

\begin{algorithm}[htp]
    \DontPrintSemicolon
    \caption{Compute Agony for a directed signed network }
    \label{alg:signed-agony}
    \KwIn{Directed siged graph $G = (V,E,w)$}
    \KwResult{Signed Agony $\alpha$ of $G$}
    Initialize $G^\prime = (V^\prime,E^\prime,w^\prime)$\;
    $V^\prime \leftarrow V$\;
    \ForEach{$e \in E$}{
        \eIf{$w(e) <0$}{         
            $e^\prime \leftarrow \left(\dest(e),\src(e)\right)$ \tcp*{Change direction of the edge}
            $E^\prime \leftarrow E^\prime \cup \{e^\prime\} $\;
            $w^\prime(e^\prime)\leftarrow -w(e)$ \tcp*{Make the weight positive}
        }
        {
            $E^\prime \leftarrow E^\prime \cup \{e\}$\;
            $w^\prime(e) \leftarrow w(e)$\;
        }
        
    }
    $\alpha \leftarrow Agony(G^\prime)$\; 
    \Return $\alpha$\;
\end{algorithm}

Now, we can compute the agony of the LSN and utilize it to predict the sign of the vote.
We follow a process similar to predicting the sign using balance theorem.
First, first assume the vote is positive and add the edge $(v,c)$ to the LSN and compute the agony and denote it $\alpha^+$.
Next, we assume the vote is negative, add the corresponding edge to the LSN and compute the agony and denote it $\alpha^-$.
If $\alpha^+ < \alpha^-$, then it means that the positive vote results in a LSN that has fewer violations of status and therefore, we can predict the vote is positive.
Similarly, we predict a negative vote if $\alpha^- < \alpha^+$.
Alternatively, we can also compute the probability of a positive vote as $p=1/(1+e^{(1-r)})$, where $r=\alpha^-/\alpha^+$.
This process is detailed in Algorithm~\ref{alg:status-pred}.

We see a directed LSN similar to the one in Figure~\ref{fig:lsn-balance}a in Figure~\ref{fig:lsn-status}.
The branches indicate the two possibilities for the vote edge $(v,c)$.
The left branch assumes that the vote is positive and adds it to the LSN.
When we transform the negative edges in the LSN we get a cycle.
As a cycle indicates that there is no hierarchy in the LSN, the agony $\alpha^+=3$ reflects the fact that each edge violates status theory.
The right branch assumes that the vote is negative and includes it in the LSN.
After transformation, we see that all the edges comply with status theory and therefore, the agony $\alpha^-=0$.
Now, the corresponding positive vote probability is $p=0$, indicating that we would predict a negative vote.
Note, this result is contradictory to the prediction made by balance theory for the same LSN.

\begin{algorithm}[htp]
    \DontPrintSemicolon
    \caption{Predict positive vote probability using status theory}
    \label{alg:status-pred}
    \KwIn{Voter $v$, Candidate $c$, Local Signed Network $LSN$}
    \KwResult{Probablity of edge $(v,c)$ being positive}
    $w_{LSN}((v,c)) \leftarrow +1$ \tcp*{Assume positive vote}
    $\alpha^+ \leftarrow SignedAgony(LSN)$\;
    $w_{LSN}((v,c)) \leftarrow -1$ \tcp*{Assume negative vote} 
    $\alpha^- \leftarrow SignedAgony(LSN)$\;
    $w_{LSN}((v,c)) \leftarrow 0$ \tcp*{Reset edge weight}
    $r \leftarrow {\alpha^-}/{\alpha^{+}}$\;
    $p \leftarrow 1/(1+e^{(1-r)})$\;
    \Return p
\end{algorithm}

\begin{figure}[!ht]
    \centering
    \tikzset{
        position/.style args={#1:#2 from #3}{
            at=(#3.#1), anchor=#1+180, shift=(#1:#2)
        }
    }

\begin{figure}[!ht]
    \centering

\begin{tikzpicture}

    \begin{scope}[every node/.style={circle,thick,draw,minimum size=9mm}]
        \node (u) at (0,0) {$u$};
        \node[position=-60:1.3cm from u] (c) {$c$};
        \node[position=-120:1.3cm from u] (v) {$v$};
        
        \node[position=-120:2cm from v] (u1)  {$u$};
        \node[position=-60:1.3cm from u1] (c1) {$c$};
        \node[position=-120:1.3cm from u1] (v1) {$v$};

        \node[position=-60:2cm from c] (u2)  {$u$};
        \node[position=-60:1.3cm from u2] (c2) {$c$};
        \node[position=-120:1.3cm from u2] (v2) {$v$};

        \node[position=-90:4cm from u1] (u3)  {$u$};
        \node[position=-60:1.3cm from u3] (c3) {$c$};
        \node[position=-120:1.3cm from u3] (v3) {$v$};

        \node[position=-90:4cm from u2] (u4)  {$u$};
        \node[position=-60:1.3cm from u4] (c4) {$c$};
        \node[position=-120:1.3cm from u4] (v4) {$v$};


    \end{scope}

    % positive edges
    \begin{scope}[
        relation/.style= {thick,draw,blue,->,>={Stealth[blue]}},
        vote/.style= {very thick,draw,red,densely dotted,->,>={Stealth[red]}},
        predict/.style={thick,draw,dashed,->,>={Stealth[black]}},
        assume/.style={thick,draw,->,>={Stealth[black]}},
        every node/.style={fill=white,rectangle,inner sep=0}]
        % \draw (v4) -- (v1) -- (v3) -- (v2) -- (v1);
        \path 
        (u) edge[vote] node[above right=1mm] {$-$} (c)  
        (v) edge[relation] node[above left=1mm] {$-$} (u)
        (v) edge[predict] (c)

        (u1) edge[vote] node[above right=1mm] {$-$} (c1)  
        (v1) edge[relation] node[above left=1mm] {$-$} (u1)
        (v1) edge[assume] node[above=1mm] {$+$} (c1)
        
        (u2) edge[vote] node[above right=1mm] {$-$} (c2)  
        (v2) edge[relation] node[above left=1mm] {$-$} (u2)
        (v2) edge[assume] node[above=1mm] {$-$} (c2)
        
        (c3) edge[vote] node[above right=1mm] {$+$} (u3)  
        (u3) edge[relation] node[above left=1mm] {$+$} (v3)
        (v3) edge[assume] node[above=1mm] {$+$} (c3)
        
        (c4) edge[vote] node[above right=1mm] {$+$} (u4)  
        (u4) edge[relation] node[above left=1mm] {$+$} (v4)
        (c4) edge[assume] node[above=1mm] {$+$} (v4)
        ;
    \end{scope}
    \begin{scope}[
        every edge/.style= {thick,draw,->},
        every node/.style={fill=white,rectangle,inner sep=0,text=black}
    ]
    \node[position=-90:4cm from u3] (a1) {$\alpha^+=3$};
    \node[position=-90:4cm from u4] (a2) {$\alpha^-=0$};
    \path     
    (v) edge[shorten >=0.5cm,shorten <=0.5cm] node[above left=0.1cm,text width=2cm,align=center] {Assume \\ positive vote} (u1)
    (c) edge[shorten >=0.5cm,shorten <=0.5cm] node[above right=0.1cm,text width=2cm,align=center] {Assume \\ Negative vote} (u2)

    ($(u1)!0.5!(u3)$) edge node[right=1mm] (convertL) {} ($(u1)!0.8!(u3)$)
    ($(u2)!0.5!(u4)$) edge node[left=1mm] (convertR) {} ($(u2)!0.8!(u4)$)
    
    (convertL) edge[draw=none] node[text width=3cm,align=center] {Transform\\ negative edges} (convertR)
    
    ($(u3)!0.6!(a1)$) edge node[right=1mm] (agonyL) {} ($(u3)!0.9!(a1)$)
    ($(u4)!0.6!(a2)$) edge node[left=1mm] (agonyR) {} ($(u4)!0.9!(a2)$)
    
    (agonyL) edge[draw=none] node[text width=3cm,align=center] {Compute\\ Agony} (agonyR)

    ;
    \end{scope}
  \end{tikzpicture}
  


    \caption{Example of LSN sign prediction using status theory.}
    \label{fig:lsn-status}
\end{figure}
    \caption{Example of LSN sign prediction using status theory.}
    \label{fig:lsn-status}
\end{figure}

\subsection{Iterative Prediction Model}

Now, Algorithm~\ref{alg:iterative-pred} describes a model that can predict the votes in a session iteratively.
We assume we have the order of votes in the session denoted by $O$ and the true votes function $w^{*}$.
Then, we create the session graph and initialize it with the candidate $c$ and the first voter as seen in line~\ref{iterative-pred:line:init-session}.
This is because, we require information from at least one prior vote to effectively predict any vote.
In most settings, such as bills in a parliament or promotion of a member in a community, there are always sponsors or nominators who propose the candidate in that session.
Therefore, we are justified in starting the session with the candidate and the first vote cast in the session graph $S$, which we assume is the nominator's vote.
Next, for each subsequent voter $v$ in the list, we add them to the session and create the LSN.
The $Predict$ function, in line~\ref{iterative-pred:line:predict} can be based on either balance theory (Algorithm~\ref{alg:balance-pred}) or status theory (Algorithm~\ref{alg:status-pred}).
After predicting the vote, we add the true vote of voter $v$ to the session $G$ and move on to the next voter in the list.
After all the votes are predicted in the session, we can update the relationship graph $H$ with the data from the current session $S$.
This can include operations such as adding nodes that were not present in $R$ and updating the edge weights based on the votes cast in the current session.

In this process we can predict all the votes in all the sessions by starting with an empty relationship graph and updating it after every session.
Therefore, the model can be compared to a "\textit{batch}" machine learning model, where each batch is a voting session and the model parameter is the relationship graph $R$.
In a batch, a machine learning model will predict an outcome based on its parameters.
Analogously, in a session, our iterative model predicts the vote using the the information gathered from the previous sessions contained in the relationship graph $R$.
After a batch is complete, the machine learning model updates its parameters based on the data in the batch.
Similarly, after the end of a session, our iterative model updates the relationships graph with voting data from the current session.

Furthermore, we can bootstrap the iterative model from a complete \textit{blank slate} where $R$ is an empty graph and then iteratively predicts sessions and updates $R$.
Therefore, the model can improve after each session by incorporating the voting information into the relationship graph.

\begin{algorithm}[htp]
    \DontPrintSemicolon
    \caption{Iterative Prediction Model}
    \label{alg:iterative-pred}
    \KwIn{Candidate $c$, Relationship graph $R=(V_R,E_R,w_R)$, Order of voters in current session $O$ and true votes $w^*$  }
    \KwResult{Predictions for current session }
    $k \leftarrow |O|$\;
    $u \leftarrow O[1]$ \tcp*{First voter} 
    $V_S \leftarrow \{c,u\}$ \tcp*{candidate and first voter}
    $E_S \leftarrow \{(u,c)\}$ \tcp*{first vote}
    $w_S((u,c)) \leftarrow w^{*}((u,c))$ \tcp*{Assign true vote}
    Initialize session graph $S = \{V_S,E_S,w_S\}$\; \label{iterative-pred:line:init-session}
    $predictions \leftarrow \emptyset$ \;
    \For{$i \leftarrow 2$ \KwTo $k$}{
        $v \leftarrow O[i]$\; 
        $V_S \leftarrow V_S \cup \{v\}$ \;
        $LSN \leftarrow S \localintersect R$\; \label{iterative-pred:line:create-lsn}
        $p \leftarrow Predict(v,c,LSN)$\; \label{iterative-pred:line:predict}
        $predictions \leftarrow predictions \cup p$\;
        $E_S \leftarrow E_S \cup \{(v,c)\}$\;
        $w_S((v,c)) \leftarrow w^{*}((v,c))$ \tcp*{Assign true vote}
    }   
    $Update(R,S)$ \tcp*{Update Relationship graph} \label{iterative-pred:line:update-relation}
    \Return $predictions$\;
\end{algorithm}

