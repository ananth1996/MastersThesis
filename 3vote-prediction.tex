\chapter{Vote Prediction}
\label{chp:vote-prediction}
In this chapter, we cover the main motivation behind predicting the vote of an individual voter and present the methods that can be used to solve this task. We discuss the contrast in perspectives that is present when predicting the result compared to predicting a vote in Section~\ref{sec:result-vs-vote}. Next, we explain the link prediction problem in signed networks and the existing approaches as well as limitations to predicting votes in Section~\ref{sec:link-prediction}. In Section~\ref{sec:linear-combination-theory}, we describe a supervised machine learning framework that can use graphs from voting and non-voting features to predict a vote. Lastly, we present our novel approach of constructing a signed graph from neighbours of the current voter and previous votes and using balance and status theory to predict the vote in Section~\ref{sec:local-signed-network}. 


\section{Result versus Vote Prediction}
\label{sec:result-vs-vote}
In this thesis, we are interested in the voting behaviour for a collective action. In such cases, members in a particular community come together as \textit{voters} to decide on a particular \textit{candidate} item. In the parliament of a government the voters are the elected members of the parliament and the candidate is usually a bill or policy matter. When it is promotion within a political party or an online community such as Wikipedia, the members vote on a candidate who has been nominated for the position. In all these cases we have two levels of decisions being made. The first the individual decisions that a voter makes with regard to the candidate. The second is the final decision that the group arrives to after everyone has voted. We refer to predicting the latter as \textit{vote prediction} and predicting the former as \textit{result prediction}. 

When the chosen problem is result prediction, we gain a macro level perspective of the incentives of a community. We can create models based on the characteristics of a candidate to predict the result of a collective action. This will lead to understanding on a communal level of what features are preferred or if there are voting blocks formed within the members based on the type of candidate. This translates to practical examples such as party level dynamics in a parliament, the topic of a bill or the credentials of a nominee \cite{burke2008mopping,yano2012textual,yogatama-etal-2011-predicting}. 

On the other hand, when we focus on the vote prediction problem, we get a deeper understanding of the dynamics between voters and the candidate. In fields such as game theory, this is well studied using frameworks such as \textit{strategic voting models} and \textit{momentum} \cite{meir2020strategic,zou2015strategicDoodle,ali2006a,banerjee1992simple}. These models are more theoretical and are studied under synthetic experiments. They still provide a foundation on which we can construct more practical models that can use external features. One such popular approach is using textual information from bills, speeches and legislature to predict votes of politicians in parliament \cite{budhwar2018predicting,gerrish2011predicting}. The next important step is to represent the voting data as networks and leveraging network features to predict and understand voter behaviour \cite{tal2015a,brito2020aBrazil,li2015voter,kearns2009behavioral}. Network analysis of voting data from a game theoretic perspective points to a similar well known model of \textit{information cascades} in social networks \cite{ali2006theory,li2015voter}.

\todo[inline]{Are the IC paras neccessary?}
Information cascades are widely studied in epidemiology where we model the spread of a contagion through infections of neighbours and the halting using a rate of recovery. Collective voting in a network can be similarly modelled as a sequence of votes spreading through a network and naturally ending when a final decision has been taken. Using an \textit{independent cascade} model we can separate the spread of information in a network into two logical problems.

The first, is a which node will be affected next. This is equivalent to knowing which node will get infected in a contagion model or which person will be next to vote in a voter model. In contagions spread this is probabilistic hence can be easily modelled. This is different for voting models as some cases might have a fixed voting order while others may not. In the cases we have a known voting order such as the roll call for voting in parliament or timestamps in promotions of Wikipedia administrators, the problem is intrinsically solved. We have mentioned other models above, that can handle cases where there is either no known order or voting happens in a non-sequential manner. 

The second problem, is how will the node will react. In the context of a contagion model, this translates to whether the now infected node becomes sick or not. This is usually a known feature of a given contagion and hence is not critical to a contagion model. In a voter mode, this problem is determining how an individual voter will cast their vote given the information of who have already cast their vote. We call this problem the \textit{independent vote prediction} problem as it considers each voter in the chain as an independent actor and only aims to predict how that particular individual would vote. In solving this problem we can gain insights into the motivations and behaviour of voters. 

The problem of independent vote prediction is studied in Karimi et al. \cite{karimi2019multicongress} and Jankowski-Lorek et al.\cite{jankowski-lorek2013MBSN}. They propose models that incorporate network information, textual data, signed graph features to predict the voting behaviour. Although their models achieve a high predictive accuracy in their respective tasks, they are quite tasks specific and cannot easily be extended to other settings. In contrast, we focus on using general signed network features of balance and status to create a model to predict the votes. 

\todo[inline,backgroundcolor=blue!25]{Add a paragraph to transition to signed link prediction}

\section{Signed Link Prediction}
\label{sec:link-prediction}


\begin{itemize}
    \item Outline of the link prediction problem for signed networks \cite{liben-nowell2007the,leskovec2010signed}
    \item How task is related to vote prediction 
    \item Balance based prediction models that use network features, matrix factorization, and triadic features \cite{leskovec2010predicting,chiang2011exploiting,agrawal2013link,gu2019link,Jiliang2015Negative,Shuang-Hong2012Friend}
    \item Status and motif based prediction models \cite{Liu2019LinkPrediction}
    \item The main limitations in directly using these methods for vote prediction task
\end{itemize}

\section{Linear Combination of Graphs}
\label{sec:linear-combination-theory}
\begin{itemize}
    \item Describe how to change the link prediction framework to suit vote prediction task
    \item Discuss how to create features by taking the local neighbours of a voters and features from graphs 
    \item Create a supervised learning problem that can be solved using linear methods
    \item How it can also incorporate signed features as additional features in prediction
    \item It can use other non-voter data specific to the task
\end{itemize}

\section{Local Signed Network}
\label{sec:local-signed-network}
\begin{itemize}
    \item Explain the concept of the local signed network for a particular user
    \item Motivate the definition with respect to voters and influence in a small network
    \item Describe how to use balance and status theory to predict the vote
    \item Balance is through creating a signed adjacency matrix and then computing the smallest eigenvalue and choosing the edge that makes the graph most balanced
    \item The status is measured using the concept of agony and in a similar way we choose the edge that has least agony when added.
    \item Use them as deterministic rules or confidence values, akin to Logistic Regression
\end{itemize}