\chapter{Vote Prediction}
\label{chp:vote-predicition}
In this chapter, we cover the main motivation behind predicting the vote of an individual voter and present the methods that can be used to solve this task. We discuss the contrast in perspectives that is present when predicting the result compared to predicting a vote in Section~\ref{sec:result-vs-vote}. Next, we explain the link prediction problem in signed networks and the existing approaches as well as limitations to predicting votes in Section~\ref{sec:link-prediction}. In Section~\ref{sec:linear-combination-theory}, we describe a supervised machine learning framework that can use graphs from voting and non-voting features to predict a vote. Lastly, we present our novel approach of constructing a signed graph from neighbours of the current voter and previous votes and using balance and status theory to predict the vote in Section~\ref{sec:local-signed-network}. 


\section{Result versus Vote Prediction}
\label{sec:result-vs-vote}
\begin{itemize}
    \item Discuss existing result predictions schemes
    \item Discuss the limitations in understanding voting dynamics through just predicting results
    \item Describe the process as an information cascade, discuss the potential Game Theory settings
    \item Show the two parts of the problem from an information cascading perspective 
     \begin{itemize}
        \item Who is going to vote next
        \item How they are going to vote
    \end{itemize}
    \item Discuss the assumptions in usual Independent Cascade (IC) models
    \item Explain the difficulty of both aspects in the domain of an election 
    \item Motivate the selection of the problem as an \textbf{Independent Vote Prediction}
\end{itemize}


\section{Signed Link Prediction}
\label{sec:link-prediction}
\begin{itemize}
    \item Discuss the existing edge predictions work
    \item Directly using signed triads as features
    \item Using triads along with network features
    \item Using user information and interaction data for predicting votes and/or elections
    \item The main drawbacks in these methods when considering an election setting
\end{itemize}

\section{Linear Combination of Graphs}
\label{sec:linear-combination-theory}
\begin{itemize}
    \item Describe the linear combination of graphs derived from user and election data
    \item Explain topic similarity, follows network, interaction networks and other features
    \item How it can also incorporate signed features as additional features in prediction
\end{itemize}

\section{Local Signed Network}
\label{sec:local-signed-network}
\begin{itemize}
    \item Explain the concept of the local signed network for a particular user
    \item Motivate the definition with respect to elections and influence
    \item Describe how to use balance and status theory to predict the vote
    \item Clarify the differences to signed edge prediction efforts
    \item Mention Agony as a way to measure status compliance here?
\end{itemize}