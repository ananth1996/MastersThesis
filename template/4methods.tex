\chapter{Methods}
\label{chapter:methods}

You have now stated your problem, and you are ready to do something
about it!  \emph{How} are you going to do that? What methods do you
use?  You also need to review existing literature to justify your
choices, meaning that why you have chosen the method to be applied in
your work.

% An example of a traditional LaTeX table
% ------------------------------------------------------------------
% A note on underfull/overfull table cells and tables:
% ------------------------------------------------------------------
% In professional typography, the width of the text in a page is always a lot
% less than the width of the page. If you are accustomed to the (too wide) text
% areas used in Microsoft Word's standard documents, the width of the text in
% this thesis layout may suprise you. However, text in a book needs wide
% margins. Narrow text is easier to read and looks nicer. Longer lines are 
% hard to read, because the start of the next line is harder to locate when
% moving from line to the next. 
% However, tables that are in the middle of the text often would require a wider
% area. By default, LaTeX will complain if you create too wide tables with
% ``overfull'' error messages, and the table will not be positioned properly
% (not centered). If at all possible, try to make the table narrow enough so
% that it fits to the same space as the text (total width = \textwidth).
% If you do need more space, you can either
% 1) ignore the LaTeX warnings 
% 2) use the textpos-package to manually position the table (read the package
%    documentation)
% 3) if you have the table as a PDF document (of correct size, A4), you can use
%    the pdfpages package to include the page. This overrides the margin
%    settings for this page and LaTeX will not complain.
% ------------------------------------------------------------------
% Another note:
% ------------------------------------------------------------------
% If your table fits to \textwidth, but the cells are so narrow that the text
% in p{..}-formatted cells does not flow nicely (you get underfull warnings 
% because LaTeX tries to justify the text in the cells) you can manually set
% the text to unjustified by using the \raggedright command for each cell 
% that you do not want to be justified (see the example below). \raggedleft 
% is also possible, of course...
% ------------------------------------------------------------------
% If you need to have linefeeds (\\) inside a cell, you must create a new
% paragraph-formatting environment inside the cell. Most common ones are 
% the minipage-environment and the \parbox command (see LaTeX documentation
% for details; or just google for ``LaTeX minipage'' and ``LaTeX parbox'').
\begin{table}
\begin{tabular}{|p{2cm}|p{3.5cm}|p{4.2cm}|p{1.8cm}|} 
% Alignment of sells: l=left, c=center, r=right. 
% If you want wrapping lines, use p{width} exact cell widths.
% If you want vertical lines between columns, write | above between the letters
% Horizontal lines are generated with the \hline command:
\hline % The line on top of the table
\textbf{Code} & \textbf{Name} & \textbf{Methods} & \textbf{Area} \\ 
\hline 
% Place a & between the columns
% In the end of the line, use two backslashes \\ to break the line,
% then place a \hline to make a horizontal line below the row 
CS-E4900 & User-Centered Methods for Product and Service Design
    & \raggedright Interviews, observations, questionnaires, probes, etc
& Usability \\ 
\hline
\multicolumn{2}{|p{6.25cm}|}{MS-E2108 Simulation (here is an example of
 multicolumn for tables)}& Details of how to build simulations &
                                                                 Computer Science \\
% The multicolumn command takes the following 3 arguments: 
% the number of cells to merge, the cell formatting for the new cell, and the
% contents of the cell
\hline
ELEC-E7130 & Internet Traffic Measurements and Analysis 
& \raggedright How to measure and analyse network
  traffic & Communi\-cations \\ \hline % here we give hint for hyphenation
\end{tabular} % for really simple tables, you can just use tabular
% You can place the caption either below (like here) or above the table
\caption{Research methodology courses}
% Place the label just after the caption to make the link work
\label{table:courses}
\end{table} % table makes a floating object with a title

If you have not yet done any (real) metholodogical courses, 
now is the time to do so or at least
check through material of suitable methodological courses. Some
methodologial courses that consentrates especially to methods in
different fields of computer science are
presented in Table~\ref{table:courses}. Remember to explain the
content of the tables (as with figures). In the table, the last column
gives the research area where the methods are often used. 

Here we used table to give an example of tables, and you can read more
about tables from the latex source file \texttt{4methods.tex}. In the
beginning of the thesis, the section Abbreviations and Acronyms is
also a long table. The difference is that longtables can continue to
next page.

